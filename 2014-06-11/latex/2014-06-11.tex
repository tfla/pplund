\documentclass[a4paper,10pt]{article}
\usepackage[T1]{fontenc}
\usepackage[utf8]{inputenc}
\usepackage[swedish]{babel}
\usepackage{setspace}
\usepackage{booktabs}
\usepackage{hyperref}                                   % Vi tycker om hyperreferenser.
\usepackage{fancyhdr}

\renewcommand{\baselinestretch}{1.4} % 1.4 PUNKTERS RADAVSTÅND

\pagestyle{fancy}
\fancyhf{}
\fancyhead[R]{\bfseries\thepage}
\fancyfoot[R]{\begin{flushright}\makebox[2in]{\hrulefill}\end{flushright} Justerare: Niklas Dahl}
\fancyfoot[L]{\begin{flushleft}\makebox[2in]{\hrulefill}\end{flushleft} Justerare: Niklas Starow}
\renewcommand{\headrulewidth}{0pt}
\renewcommand{\footrulewidth}{0pt}
\addtolength{\headheight}{0pt}
\fancypagestyle{plain}{
    \fancyhead{} % get rid of headers on plain pages
    \renewcommand{\headrulewidth}{0pt} % and the line
}

\hypersetup{
  pdfborder={0 0 0},         % Ingen "ram".
  pdftitle={Mötesprotokoll}, % Titlen.
  pdfauthor={Timmy Larsson}, % Författaren.
  colorlinks=false,          % false: boxed links; true: colored links
  linkcolor=red,             % color of internal links
  citecolor=green,           % color of links to bibliography
  filecolor=magenta,         % color of file links
  urlcolor=cyan              % color of external links
}



%\textwidth=6.5in        %
%\textheight=9.0in       %
%\headsep=0.25in         %


\title{\vspace{-1.5in}\textmd{\textbf{Mötesprotokoll Styrelsemöte Piratpartiet Lund}}}
\date{11 juni 2014}
\author{}


\begin{document}
\maketitle

\section{Mötets Öppnande}
Timmy Larsson förklarade mötet öppnat kl 17:06.

\section{Fastställande av röstlängd}
Röstlängden fastställdes till:
\begin{enumerate}
\item Dan Persson
\item David Boholm
\item Wilhelm Dahl
\item Jan-Erik Malmqvist
\item Andreas Söderlund
\item Timmy Larsson
\end{enumerate}

Timmy yrkar på att alla närvarande adjungeras med yttranderätt och förslagsrätt. \\
Mötet biföll yrkandet

\section{Fastställande av dagordning}
Mötet beslutade att fastställa dagordningen som utskickad med den övriga frågan ''Fråga om listan till kommunfullmäkrigevalet'' tillagd.

\section{Mötets beslutsmässighet}
Mötet ansåg sig behörigt kallat.

\newpage

\section{Val av mötesordförande}
Timmy Larsson föreslog sig själv till mötesordförande.\\
Mötet beslutade att välja Timmy Larsson till mötesordförande.

\section{Val av mötessekreterare}
Timmy Larsson föreslog Andreas Söderlund till mötessekreterare.\\
Mötet beslutade att välja Andreas Söderlund till mötessekreterare

\section{Val av två personer att justera protokollet}
Timmy Larsson nominerade Niklas Starow och Niklas Dahl till justerare.\\
Mötet beslutade att välja Niklas Starow och Niklas Dahl till justerare.

\section{Utvärdering av valkampanjen}
David anförde angående kampanjen inför EU-valet och positiva/negativa aspekter diskuterades.

\section{Planering inför höstens val}
Strategi diskuterades. Punktmarkering. Översättning av kampanjmaterial. \\
Andreas Söderlund yrkade på att välja David Boholm till strategiansvarig i Lunds kommun. \\
Mötet beslutade att bifalla yrkandet.

\subsection*{i) Valmanifest}
Utkastet till valmanifest gicks igenom och ändringar föreslogs. Manifestet kommer att diskuteras vidare på onsdag den 17/6. \\

\subsection*{ii) Kampanjmaterial}
Varierande för olika stadsdelar, olika språk, olika budskap. Tre huvudbudskap. Skriv ihop sammanfattningar på 5-10 punkter för de olika fokusgrupperna.

\subsection*{iii) Valvaka}
Frågan bordlades till nästa möte.

\subsection{iv) Arbetsgrupper}
Frågan bordlades till nästa möte.

\section{Övriga frågor}
\subsection*{Fråga om listan till kommunfullmäktigevalet}
David anförde om läget inför höstens val. Två avhopp från listan har anmälts.


\section{Mötets avslutande}
Timmy Larsson förklarade mötet för avslutat kl 23:06.

\end{document}
