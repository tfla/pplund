\documentclass[a4paper,10pt]{article}
\usepackage[T1]{fontenc}
\usepackage[utf8]{inputenc}
\usepackage[swedish]{babel}
\usepackage{setspace}
\usepackage{booktabs}
\usepackage{hyperref}                                   % Vi tycker om hyperreferenser.
\usepackage{fancyhdr}

\renewcommand{\baselinestretch}{1.4} % 1.4 PUNKTERS RADAVSTÅND

\pagestyle{fancy}
\fancyhf{}
\fancyhead[R]{\bfseries\thepage}
\fancyfoot[R]{\begin{flushright}\makebox[2in]{\hrulefill}\end{flushright} Mötesordförande: Timmy Larsson \begin{flushright}\makebox[2in]{\hrulefill}\end{flushright} Mötessekreterare: Andreas Söderlund}
\fancyfoot[L]{\begin{flushleft}\makebox[2in]{\hrulefill}\end{flushleft} Justerare: David Boholm \begin{flushleft}\makebox[2in]{\hrulefill}\end{flushleft} Justerare: Wilhelm Dahl}
\renewcommand{\headrulewidth}{0pt}
\renewcommand{\footrulewidth}{0pt}
\addtolength{\headheight}{0pt}
\fancypagestyle{plain}{
    \fancyhead{} % get rid of headers on plain pages
    \renewcommand{\headrulewidth}{0pt} % and the line
}

\hypersetup{
  pdfborder={0 0 0},         % Ingen "ram".
  pdftitle={Mötesprotokoll}, % Titlen.
  pdfauthor={Timmy Larsson}, % Författaren.
  colorlinks=false,          % false: boxed links; true: colored links
  linkcolor=red,             % color of internal links
  citecolor=green,           % color of links to bibliography
  filecolor=magenta,         % color of file links
  urlcolor=cyan              % color of external links
}



%\textwidth=6.5in        %
%\textheight=9.0in       %
%\headsep=0.25in         %


\title{\vspace{-1.5in}\textmd{\textbf{Mötesprotokoll Styrelsemöte Piratpartiet Lund}}}
\date{27 april 2014}
\author{}


\begin{document}
\maketitle

\section{Mötets Öppnande}
Timmy Larsson förklarade mötet öppnat kl 16:22.

\section{Fastställande av röstlängd}
Röstlängden fastställdes till:
\begin{enumerate}
\item Andreas Söderlund
\item Timmy Larsson
\item David Boholm
\item Wilhelm Dahl
\item Jan-Erik Malmkvist
\end{enumerate}

\section{Fastställande av dagordning}
Mötet beslutade att flytta upp punkten valvaka till punkt 6 samt att lägga till punkten "lägga till medlem" under övriga punkter. Utöver det godkänndes dagordningen som den står.

\section{Mötets beslutsmässighet}
Kallelsen skickades ut 2014-04-20 och mötet fanns beslutsmässig

\newpage

\section{Val av mötesordförande}
Mötet valde Timmy Larsson till mötesordförande

\section{Val av mötessekreterare}
Mötet valde Andreas Söderlund till mötessekreterare

\section{Val av två personer att justera protokollet}
David nominerar sig själv till justerare.\\
Timmy nominerar Ville till justerare.\\
Mötet godkännde dessa.

\section{Valvaka}
Mötet gav i uppdrag till Jan-Erik att kolla upp loftet.\\
Mötet gav i uppdrag till David att kolla upp Estniska huset.\\
Mötet gav i uppdrag till David och Jan-Erik att besluta om valvakan.

\section{Valstuga}
Beslutet om tillståndet fördröjt.

\section{Valmanifest}
Arbetet fortskrider.\\
Mötet beslutade att bordlägga frågan.

\section{Övriga frågor}
\subsection*{Ansökan om medlemsskap i Piratpartiet Lund}
Mattias Bjärnemalm söker om medlemsskap i Piratpartiet Lund enl. paragraf 2 i stadgarna.\\
Mötet beslutade att godkänna hans medlemsskapsansökan.

\section{Mötets avslutande}
Timmy Larsson förklarade mötet för avslutat kl 17:02.

\end{document}