\documentclass[a4paper,10pt]{article}
\usepackage[T1]{fontenc}
\usepackage[utf8]{inputenc}
\usepackage[swedish]{babel}
\usepackage{setspace}
\usepackage{booktabs}
\usepackage{hyperref}                                   % Vi tycker om hyperreferenser.
\usepackage{fancyhdr}

\renewcommand{\baselinestretch}{1.4} % 1.4 PUNKTERS RADAVSTÅND

\pagestyle{fancy}
\fancyhf{}
\fancyhead[R]{\bfseries\thepage}
\fancyfoot[R]{\begin{flushright}\makebox[2in]{\hrulefill}\end{flushright} Justerare: Jan-Erik Malmquist}
\fancyfoot[L]{\begin{flushleft}\makebox[2in]{\hrulefill}\end{flushleft} Justerare: Dan Persson}
\renewcommand{\headrulewidth}{0pt}
\renewcommand{\footrulewidth}{0pt}
\addtolength{\headheight}{0pt}
\fancypagestyle{plain}{
    \fancyhead{} % get rid of headers on plain pages
    \renewcommand{\headrulewidth}{0pt} % and the line
}

\hypersetup{
  pdfborder={0 0 0},         % Ingen "ram".
  pdftitle={Mötesprotokoll}, % Titlen.
  pdfauthor={Timmy Larsson}, % Författaren.
  colorlinks=false,          % false: boxed links; true: colored links
  linkcolor=red,             % color of internal links
  citecolor=green,           % color of links to bibliography
  filecolor=magenta,         % color of file links
  urlcolor=cyan              % color of external links
}



%\textwidth=6.5in        %
%\textheight=9.0in       %
%\headsep=0.25in         %


\title{\vspace{-1.5in}\textmd{\textbf{Mötesprotokoll styrelsemöte Piratpartiet Lund}}}
\date{\today}
\author{}


\begin{document}
\maketitle

\section{Mötets Öppnande}
Timmy Larsson föklarade mötet öppnat kl 14:14.

\section{Fastställande av röstlängd}
Röstlängden fastställdes till:
\begin{enumerate}
\item Timmy Larsson
\item Jan-Erik Malmquist
\item Dan Persson
\item David Boholm
\item Andreas Söderlund
\item Wilhelm Dahl
\item Gabriela Galvao
\end{enumerate}

\section{Fastställande av dagordning}
Timmy yrkade på att slå samman mötespunkterna 'val av mötesordförande', 'val av mötessekreterare' och 'val av två personer att justera protokollet' till en punkt 'val av mötets ordförande, sekreterare och två justerare'.
Mötet fastställde röstlängden som den låg med den föreslagna ändringen.

\section{Mötets beslutsmässighet}
Mötet ansåg sig vara behörigt då kallelsen skickats ut minst två dagar i förväg och 2/3 av styrelsen var närvarande.

\section{Val av mötets ordförande, sekreterare och två justerare}
Mötet fastställde mötesfunkionärerna till:
\begin{itemize}
\item Ordförande: Timmy Larsson
\item Sekreterare: Andreas Söderlund
\item Justerare: Jan-Erik Malmquist, Dan Persson
\end{itemize}

\section{Godkännande av dagordning}
Mötet godkände dagordningen som den skickats ut i kallelsen.

\section{Fråga om firmateckningsrätt}
Timmy yrkade på att Ordförande (Timmy Larsson) och Kassör (Jan-Erik Malmquist) skulle fortsätta ha firmateckningsrätt för lokalavdelningen.

\section{Genomgång av verksamhetsplan och budget samt fördelning av ansvarsposter}
%David har ansökt om att ha en valstuga. Vi vill ha det klart till den 26:e april då vi vill sätta ihop den och liknande. Bemanning tar vi och bestämmer om när vi vet exakta tider. Jan-Erik tar ansvar ang. friggeboden. Vi kommer troligtvis köpa en byggsats.\\
Timmy nominerar sig själv och Jan-Erik till att ta hand om valstuga.\\
David nominerar sig själv till att ta ansvar för allt som gäller administration av valstugan.\\
David nominerar alla i styrelsen för att kolla var vi kan lägga valvakan. Ansvarig beslutas nästa styrelsemöte.\\
David nominerar sig själv till att ringa runt (även till Estniska huset) ang. valvaka.\\
Mötet beslutade att bifalla yrkandena.

\section{Fråga om valstrategisk planering}
Timmy yrkar på att mötet tar up Fråga om regelbundna styrelsemöten innan Fråga om valstrategisk planering.\\
Mötet beslutade att bifalla yrkandet.
%vi kommer att sätta upp affischer, prata på skolor, kanske sätta upp affischer på personkandidater. Om man vill affischera så pratar man med MAB eller med David. Dörrknackning, är det kanste redan för sent? Vi måste maximera antalet vi når ut till och då är inte dörrknackning det mest effektiva. Vi måste höra av oss till partiets medlemmar i Lund och fråga dem om att vara aktiva under valet. Rundringning är nog det bästa. Vaktmästarkoden för AFB:s bostäder är 01E.\\
Mötet ålägger Dan och Niklas Starov till att ringa runt piratpartiets medlemmar i Lund för att fråga dem om de kan tänka sig att engagera sig inför valen.\\
%Vi fick en notis i torsdagens sydsvenska där det står vilka som är top fem på piratpartiets kommunlista i lund. Hur ska vi göra med info på radio och tidningar? Kanske man kan göra reklam på spotify.
Gabriela och David håller i kontakten med media.\\
%Personlig kampanj inför EU-valet. Kandiaterna från Lund är MAB och Niklas Dahl. Ska vi göra liknande i Lund? Niklas vill inte men det vill vi andra :) Ska vi ha vägplanscher också? Kanske kan bli svårt med pengar och tillstånd.
Ville tar på sig att designa affischerna på Niklas och MAB. Dan kollar upp tryckerier. David kollar med mediatryck ang. upptryckning.\\
%Ville behöver relevant material! Format är A3 eller 50\times70
%Utdelning av flyers. Nu delas det ut klockan tre vid stationen på måndag-fredag. Bör det delas ut klockan fem också? Det är många som pendlar vid den tiden. Rekomendation om Skånes sms-grupp. Skolor diskuteras också. Anna Troberg kommer vara i Lund och dela ut flyers klockan tre på tisdag den 15:e april.
%Möte ang. valmanifestet kanske ska läggas efter EU-valet? Vi sätter upp https://piratenpad.de/p/kommentarermanifest för att lägga in kommentarer på det valmanifest som Dahl, Dahl och Galvao arbetar på.
Löpande rapportering ang. utvecklingen av valmanifestet 

\section{Fråga om regelbundna styrelsemöten}
Förslag på att ha styrelsemöte söndagen den 27:e april klockan 16:00, söndagen den 11:e maj klockan 16:00 samt tisdagen den 20:e maj klockan 18:00.\\
Mötet beslutade att bifalla förslaget.\\
Timmy nominerade David till att vara mötesordförande.\\
Mötet beslutade att bifalla yrkandet.

\section{Övriga frågor}
Då det inte lyfts några övriga frågor ansåg mötet punkten behandlad.

\section{Mötets avslutande.}
Timmy förklarade mötet avslutat kl 16:20.

\end{document}
