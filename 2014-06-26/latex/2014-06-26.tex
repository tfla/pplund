\documentclass[a4paper,10pt]{article}
\usepackage[T1]{fontenc}
\usepackage[utf8]{inputenc}
\usepackage[swedish]{babel}
\usepackage{setspace}
\usepackage{booktabs}
\usepackage{hyperref}                                   % Vi tycker om hyperreferenser.
\usepackage{fancyhdr}

\renewcommand{\baselinestretch}{1.4} % 1.4 PUNKTERS RADAVSTÅND

\pagestyle{fancy}
\fancyhf{}
\fancyhead[R]{\bfseries\thepage}
\fancyfoot[R]{\begin{flushright}\makebox[2in]{\hrulefill}\end{flushright} Justerare: Wilhelm Dahl}
\fancyfoot[L]{\begin{flushleft}\makebox[2in]{\hrulefill}\end{flushleft} Justerare: Dan Persson}
\renewcommand{\headrulewidth}{0pt}
\renewcommand{\footrulewidth}{0pt}
\addtolength{\headheight}{0pt}
\fancypagestyle{plain}{
    \fancyhead{} % get rid of headers on plain pages
    \renewcommand{\headrulewidth}{0pt} % and the line
}

\hypersetup{
  pdfborder={0 0 0},         % Ingen "ram".
  pdftitle={Mötesprotokoll}, % Titlen.
  pdfauthor={Timmy Larsson}, % Författaren.
  colorlinks=false,          % false: boxed links; true: colored links
  linkcolor=red,             % color of internal links
  citecolor=green,           % color of links to bibliography
  filecolor=magenta,         % color of file links
  urlcolor=cyan              % color of external links
}



%\textwidth=6.5in        %
%\textheight=9.0in       %
%\headsep=0.25in         %


\title{\vspace{-1.5in}\textmd{\textbf{Mötesprotokoll Styrelsemöte Piratpartiet Lund}}}
\date{26 juni 2014}
\author{}


\begin{document}
\maketitle

\section{Mötets Öppnande}
Timmy Larsson förklarade mötet öppnat kl 17:01.

\section{Fastställande av röstlängd}
Röstlängden fastställdes till:
\begin{enumerate}
\item Dan Persson
\item Wilhelm Dahl
\item Jan-Erik Malmqvist
\item Timmy Larsson
\item Andreas Söderlund
\end{enumerate}
Mötet beslutade att adjungera alla närvarande med yttrande och förslagsrätt.

\section{Fastställande av dagordning}
Mötet beslutade att fastställa dagordningen som den låg i kallelsen med den övriga frågan ''Nästkommande styrelsemöte'' tillagd.

\section{Mötets beslutsmässighet}
Mötet ansåg sig behörigt kallat.

\newpage

\section{Val av mötesordförande}
Timmy Larsson föreslog Jan-Erik Malmquist till mötesordförande.\\
Mötet beslutade att välja Jan-Erik Malmquist till mötesordförande.

\section{Val av mötessekreterare}
Timmy Larsson föreslog sig själv till mötessekreterare.\\
Mötet beslutade att välja Timmy Larsson till mötessekreterare

\section{Val av två personer att justera protokollet}
Timmy Larsson nominerade Dan Persson och Wilhelm Dahl till justerare.\\
Mötet beslutade att välja Dan Persson och Wilhelm Dahl till justerare.

\section{Valmanifest}
Föreslagna ändringar har genomförts i enlighet med förra mötets uppmaningar. \\
Valmanifestet gicks igenom och vissa punkter fick förfinade formuleringar.

\section{Kampanjmaterial}
Styrelsen uppmärksammades på att beställning skett i enlighet med per capsulam-beslut taget måndagen den 23/6 2014 med undantag för A1-affischerna som visade sig vara nästan tre gånger så dyra som uppskattat. Styrelsen fann inga problem med detta. \\
Valmanifestet gicks igenom för att hitta vilka punkter som påverkar vilka målgrupper. \\
Flygbladens grundläggande utformning spikades med tre huvudområden per A65-flygblad och punkter som bör tas upp valdes. \\
På A5-flygbladen kommer fyra rubriker användas och så många av punkterna som diskuterades under mötet som möjligt bör användas.

\section{Valvaka}
Styrelsen är positiv till att ha valvaka på Stamstället igen.

\section{Arbetsgrupper}
Frågan bordlades till nästa möte.

\section{Övriga frågor}
\subsection*{Fråga om nästkommande styrelsemöte}
Två styrelsemötens datum och tid fastställdes, de kommer att inträffa 2014-07-02 kl 17:00 och 2014-07-06 kl 16:00.

\section{Mötets avslutande}
Timmy Larsson förklarade mötet för avslutat kl 19:56.

\end{document}
