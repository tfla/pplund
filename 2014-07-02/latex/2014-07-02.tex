\documentclass[a4paper,10pt]{article}
\usepackage[T1]{fontenc}
\usepackage[utf8]{inputenc}
\usepackage[swedish]{babel}
\usepackage{setspace}
\usepackage{booktabs}
\usepackage{hyperref}                                   % Vi tycker om hyperreferenser.
\usepackage{fancyhdr}

\renewcommand{\baselinestretch}{1.4} % 1.4 PUNKTERS RADAVSTÅND

\pagestyle{fancy}
\fancyhf{}
\fancyhead[R]{\bfseries\thepage}
\fancyfoot[R]{\begin{flushright}\makebox[2in]{\hrulefill}\end{flushright} Justerare: Wilhelm Dahl}
\fancyfoot[L]{\begin{flushleft}\makebox[2in]{\hrulefill}\end{flushleft} Justerare: Dan Persson}
\renewcommand{\headrulewidth}{0pt}
\renewcommand{\footrulewidth}{0pt}
\addtolength{\headheight}{0pt}
\fancypagestyle{plain}{
    \fancyhead{} % get rid of headers on plain pages
    \renewcommand{\headrulewidth}{0pt} % and the line
}

\hypersetup{
  pdfborder={0 0 0},         % Ingen "ram".
  pdftitle={Mötesprotokoll}, % Titlen.
  pdfauthor={Timmy Larsson}, % Författaren.
  colorlinks=false,          % false: boxed links; true: colored links
  linkcolor=red,             % color of internal links
  citecolor=green,           % color of links to bibliography
  filecolor=magenta,         % color of file links
  urlcolor=cyan              % color of external links
}



%\textwidth=6.5in        %
%\textheight=9.0in       %
%\headsep=0.25in         %


\title{\vspace{-1.5in}\textmd{\textbf{Mötesprotokoll Styrelsemöte Piratpartiet Lund}}}
\date{2 juli 2014}
\author{}


\begin{document}
\maketitle

\section{Mötets Öppnande}
Timmy Larsson förklarade mötet öppnat kl 17:10.

\section{Fastställande av röstlängd}
Röstlängden fastställdes till:
\begin{enumerate}
\item Dan Persson
\item Wilhelm Dahl
\item Jan-Erik Malmqvist
\item Timmy Larsson
\end{enumerate}
Alla närvarande adjungerades med yttrande- och förslagsrätt.

\section{Fastställande av dagordning}
Mötet beslutade att fastställa dagordningen som den låg i kallelsen.

\section{Mötets beslutsmässighet}
Mötet ansåg sig behörigt kallat.

\newpage

\section{Val av mötesordförande}
Timmy Larsson föreslog sig själv till mötesordförande.\\
Mötet beslutade att välja Timmy Larsson till mötesordförande.

\section{Val av mötessekreterare}
Timmy Larsson föreslog Niklas Bolmdahl till mötessekreterare.\\
Mötet beslutade att välja Niklas Bolmdahl till mötessekreterare

\section{Val av två personer att justera protokollet}
Timmy Larsson nominerade Dan Persson och Wilhelm Dahl till justerare.\\
Mötet beslutade att välja Dan Persson och Wilhelm Dahl till justerare.

\section{Valmanifest}
Mötet konstaterade att det inte finns något att tillägga i frågan för tillfället och att arbetet med valmanifestet bör skjutas upp till förmån för kampanjmaterialet som måste vara färdigt 2014-07-09.

\section{Kampanjmaterial}
Wilhelm publicerar löpande materialet på sin hemsida. \\
Mötet angav diverse språkliga och grammatiska ändringar till fördjupningsfoldern, även formuleringar samt vissa argument blev specificerade/ändrade. \\
Mötet bestämde att fortsätta arbetet med brainstorming via mail. \\
Bilder till flyers/foldrar/affischer behöver fotas/lämnas in till Wilhelm absolut senast söndag, innan mötet som hålls den dagen. \\
Förslag till slogans för specifika folders:
\begin{itemize}
\item Nu ska Sveriges bästa studentstad bli bättre.
\item Integration - Mer än medborgarskap
\end{itemize}
Affischer: \\
Personer på affischer är Timmy, Carolina, Wilhelm och Jan-Erik. \\
\begin{itemize}
\item Timmys budskap: Ett mer öppet Lund.
\item Wilhelms budskap: Ta emot fler invandrare i Lund
\item Carolinas budskap: --inhämta från Carolina (ej närvarande vid mötet)
\item Jan-Eriks budksap: --inhämta från Jan-Erik (ej närvarande vid mötet)
\end{itemize}
Slogans och foton har samma deadline, dvs senast söndag.

\section{Valvaka}
Frågan bordlades till nästa möte.

\section{Arbetsgrupper}
Frågan bordlades till nästa möte.

\section{Övriga frågor}
\subsection*{Homeparties}
Niklas Starow har tillsammans med Amelia Andersdotter börjat ta fram material för utdelning via direktreklam bland annat i Lund. Materialet delades ut och ska gås igenom av alla närvarande för återkoppling samt idéer.

\section{Mötets avslutande}
Timmy Larsson förklarade mötet för avslutat kl 20:31.

\end{document}
