\documentclass[a4paper,10pt]{article}
\usepackage[T1]{fontenc}
\usepackage[utf8]{inputenc}
\usepackage[swedish]{babel}
\usepackage{setspace}
\usepackage{booktabs}
\usepackage{hyperref}                                   % Vi tycker om hyperreferenser.
\usepackage{fancyhdr}

\renewcommand{\baselinestretch}{1.4} % 1.4 PUNKTERS RADAVSTÅND

\pagestyle{fancy}
\fancyhf{}
\fancyhead[R]{\bfseries\thepage}
\fancyfoot[R]{\begin{flushright}\makebox[2in]{\hrulefill}\end{flushright} Justerare: Niklas Starow}
\fancyfoot[L]{\begin{flushleft}\makebox[2in]{\hrulefill}\end{flushleft} Justerare: Dan Persson}
\renewcommand{\headrulewidth}{0pt}
\renewcommand{\footrulewidth}{0pt}
\addtolength{\headheight}{0pt}
\fancypagestyle{plain}{
    \fancyhead{} % get rid of headers on plain pages
    \renewcommand{\headrulewidth}{0pt} % and the line
}

\hypersetup{
  pdfborder={0 0 0},         % Ingen "ram".
  pdftitle={Mötesprotokoll}, % Titlen.
  pdfauthor={Timmy Larsson}, % Författaren.
  colorlinks=false,          % false: boxed links; true: colored links
  linkcolor=red,             % color of internal links
  citecolor=green,           % color of links to bibliography
  filecolor=magenta,         % color of file links
  urlcolor=cyan              % color of external links
}



%\textwidth=6.5in        %
%\textheight=9.0in       %
%\headsep=0.25in         %


\title{\vspace{-1.5in}\textmd{\textbf{Mötesprotokoll Styrelsemöte Piratpartiet Lund}}}
\date{18 juni 2014}
\author{}


\begin{document}
\maketitle

\section{Mötets Öppnande}
Timmy Larsson förklarade mötet öppnat kl 17:15.

\section{Fastställande av röstlängd}
Röstlängden fastställdes till:
\begin{enumerate}
\item Dan Persson
\item Wilhelm Dahl
\item Jan-Erik Malmqvist
\item Timmy Larsson
\end{enumerate}

\section{Fastställande av dagordning}
Mötet beslutade att fastställa dagordningen som den låg i kallelsen med den övriga frågan ''Nästkommande styrelsemöte'' tillagd.

\section{Mötets beslutsmässighet}
Mötet ansåg sig behörigt kallat.

\newpage

\section{Val av mötesordförande}
Timmy Larsson föreslog Niklas Dahl till mötesordförande.\\
Mötet beslutade att välja Niklas Dahl till mötesordförande.

\section{Val av mötessekreterare}
Timmy Larsson föreslog sig själv till mötessekreterare.\\
Mötet beslutade att välja Timmy Larsson till mötessekreterare

\section{Val av två personer att justera protokollet}
Timmy Larsson nominerade Dan Persson och Niklas Starow till justerare.\\
Mötet beslutade att välja Dan Persson och Niklas Starow till justerare.

\section{Valmanifest}
Ordningen för rubrikerna och sakfrågorna fastställdes rekursivt.

\section{Kampanjmaterial}
Följande preliminära beställning lades fram, och skall beslutas enligt per-capsulam när finansiering säkerställts.
Affischer:
\begin{center}
    \begin{tabular}{ | l | l |}
    \hline
	\textbf{Antal (storlek)} & \textbf{Pris} \\ \hline
    250*4 (B1)               & 4*1757.20kr \\ \hline
	1000*4 (A2)              & 4*2794.02kr \\ \hline
	5000*4 (A3)              & 4*(ca 1000)kr$^{1}$ \\
	\hline
    \end{tabular} \\
	\textit{1: Uppskattning, då prisuppgift saknas i piratshoppen.}
\end{center}

Flygblad:
\begin{center}
	\begin{tabular}{ | l | l |}
	\hline
	\textbf{Antal (storlek)} & \textbf{Pris} \\ \hline
	2x20000 (A65)            & 2*1699.10kr \\ \hline
    20000 (A5)               & 2410.40kr \\
	\hline
	\end{tabular}
\end{center}

Foldrar:
\begin{center}
	\begin{tabular}{ | l | l |}
	\hline
	\textbf{Antal (stolek)} & \textbf{Pris} \\ \hline
    10000 (8-sidiga A5)     & 8240.80kr \\
	\hline
	\end{tabular}
\end{center}

\section{Valvaka}
Frågan bordlades till nästa möte.

\section{Arbetsgrupper}
Frågan bordlades till nästa möte.

\section{Övriga frågor}
\subsection*{Fråga om nästkommande styrelsemöte}
Datum och tid för nästa möte fastställdes till den 24/6 kl 17:00.

\section{Mötets avslutande}
Timmy Larsson förklarade mötet för avslutat kl 22:50.

\end{document}
