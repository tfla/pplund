\documentclass[a4paper,10pt]{article}
\usepackage[T1]{fontenc}
\usepackage[utf8]{inputenc}
%\usepackage[swedish]{babel}
\usepackage{setspace}
\usepackage{booktabs}
\usepackage{hyperref}                                   % Vi tycker om hyperreferenser.
\usepackage{fancyhdr}

\renewcommand{\baselinestretch}{1.4} % 1.4 PUNKTERS RADAVSTÅND

\pagestyle{fancy}
\fancyhf{}
\fancyhead[R]{\bfseries\thepage}
\fancyfoot[R]{\begin{flushright}\makebox[2in]{\hrulefill}\end{flushright} Mötesordförande: Timmy Larsson \begin{flushright}\makebox[2in]{\hrulefill}\end{flushright} Mötessekreterare: Andreas Söderlund}
\fancyfoot[L]{\begin{flushleft}\makebox[2in]{\hrulefill}\end{flushleft} Justerare: Dan Persson\begin{flushleft}\makebox[2in]{\hrulefill}\end{flushleft} Justerare: Wilhelm Dahl}
\renewcommand{\headrulewidth}{0pt}
\renewcommand{\footrulewidth}{0pt}
\addtolength{\headheight}{0pt}
\fancypagestyle{plain}{
    \fancyhead{} % get rid of headers on plain pages
    \renewcommand{\headrulewidth}{0pt} % and the line
}

\hypersetup{
  pdfborder={0 0 0},         % Ingen "ram".
  pdftitle={Mötesprotokoll}, % Titlen.
  pdfauthor={Timmy Larsson}, % Författaren.
  colorlinks=false,          % false: boxed links; true: colored links
  linkcolor=red,             % color of internal links
  citecolor=green,           % color of links to bibliography
  filecolor=magenta,         % color of file links
  urlcolor=cyan              % color of external links
}



%\textwidth=6.5in        %
%\textheight=9.0in       %
%\headsep=0.25in         %


\title{\vspace{-1.5in}\textmd{\textbf{Mötesprotokoll Styrelsemöte Piratpartiet Lund}}}
\date{2014-12-17}
\author{}


\begin{document}
\maketitle

\section{Mötets Öppnande}
Timmy Larsson förklarade mötet öppnat kl 16:00.

\section{Fastställande av röstlängd}
Röstlängden fastställdes till:
\begin{enumerate}
\item Dan Persson
\item Wilhelm Dahl
\item Jan-Erik Malmqvist
\item Andreas Söderlund
\item Timmy Larsson
\end{enumerate}

\section{Mötets beslutsmässighet}
Mötet ansåg att de hade beslutsmässighet.

\section{Fastställande av dagordning}
Mötet beslutade att fastställa dagordningen som i kallelsen

\newpage

\section{Val av mötesordförande}
Timmy Larsson föreslog sig själv till mötesordförande.\\
Mötet beslutade att välja Timmy Larsson till mötesordförande.

\section{Val av mötessekreterare}
Timmy Larsson föreslog Andreas Söderlund till mötessekreterare.\\
Mötet beslutade att välja Andreas Söderlund till mötessekreterare

\section{Val av två personer att justera protokollet}
Timmy Larsson nominerade Dan Persson och Wilhelm Dahl till justerare.\\
Mötet beslutade att välja Dan Persson och Wilhelm Dahl till justerare.

\section{Diskussion kring nyvalet}
Nyvalet diskuterades. Mötet noterade att det finns ett nytt piratforum 
\begin{verbatim}
https://medlem.piratpartiet.se/forum/
\end{verbatim}
Mötet beslutade att ansöka om att ha en valstuga i Lund.

\section{Diskussion kring PL:s avgång}
Partiledningens avgång diskuterades.

\section{Handlingsplan}
Handlingsplan diskuterades.\\
Mötet beslutade att avvakta för beslut uppifrån.

\section{Övriga frågor}
Mötet beslutade att kalla till årsmöte lördagen den 31:a januari 2015.

\section{Mötets avslutande}
Timmy Larsson förklarade mötet för avslutat kl 17:17.

\end{document}
